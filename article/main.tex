%% V1.0
%% by Gabriel Garcia, gabrcg@gmail.com
%% This is a template for Udacity projects using IEEEtran.cls

%% Be Udacious!


\documentclass[10pt,journal,compsoc]{IEEEtran}

\usepackage[pdftex]{graphicx}    
\usepackage{cite}
\usepackage{listings}
\hyphenation{op-tical net-works semi-conduc-tor}

\newcommand{\subsubsubsection}[1]{\paragraph{#1}\mbox{}\\}

\setcounter{secnumdepth}{4}
\setcounter{tocdepth}{4}

\begin{document}

\title{Map My World Project}



\author{Reinaldo Ossuna}

\markboth{Map My World project, Robotics Nanodegree Program, Udacity}%
{}
\IEEEtitleabstractindextext{%

% TODO
\begin{abstract}
\end{abstract}

% Note that keywords are not normally used for peerreview papers.
\begin{IEEEkeywords}
Robot, Udacity, SLAM, EKFSLAM, GraphSLAM, RTAB-MAP
\end{IEEEkeywords}}

\maketitle
\IEEEdisplaynontitleabstractindextext
\IEEEpeerreviewmaketitle
\section{Introduction}
\label{sec:introduction}

\IEEEPARstart{S}{LAM} is the process by which the system performace the simultaneous estimation of the state of the
robot equiped with on-board sensors and the construction of the map. This can be called the egg-chicken problem because
normaly the robot need the map to performace the localization problem and the pose is needed to performace the
mapping.\\
A solution to the SLAM problem has been seen as a `Holly grail' for the mobile robotic community as it would provide the
means to make a robot truly autonomous.\ref{Bailey2006}


\section{Background}

Almong all methods proposed in the literature to solve the SLAM algoriths, we could point out some facts.

\subsection{Occupancy grid}

Occupancy grid maps address the problem of generating consistent maps from noisy and uncertain measurement data, under
the assumption that the robot pose is known.\\
The standard of the algorithm is to calculate the posterior over maps given $z_{1:t}$ the set of measurements up to time
$t$ and the $x_{1:t}$ path of the robot.

\begin{equation}
  p(m|z_{1:t},x_{1:t})
\end{equation}

The occupancy mapping algorithm estimates for each grid cell individually the posterior probability of occupancy. It is
an adaptation of the binary Bayes filter for static environments.\\
The algorithm loops through all grid cells $i$, and updates those that
fall into the sensor conde of the measuremets $zt$. For those where it does, it updates the occupancy value by the
function inverse\_sensor\_model().


Data from multiple sensors can be fused into a single map. By maintaing multiple maps, one for each sensor and
extracting the most pessimistic occupancy value when making navigation decisions. This procedure is preferable when
differents sensors are sensitive to differents types of obstacles.

\subsection{online SLAM}

Estimate the posterior over the momentary pose along with the map:
\begin{equation}
  p(x_t, m| z_{1:t},u_{1:t})
\end{equation}

We can notice that this algorith involves estimating the \textit{actual pose} and the map.
\subsection{Full SLAM}
Estimate the posterior over the \textit{entire path} $x_{1:t}$ along with the map.

\begin{equation}
  p(x_{1:t}, m| z_{1:t},u_{1:t})
\end{equation}

\subsection{EKF SLAM}

The Extended Kalman Filter (EKF) can be viewed as a variant of a Bayesian Filter. It's provide a recursive estimate of
the state of dynamic system.\\
The EKF uses a prediction and an update step. The predictions step calculates the probability of the current state
$x_t$, while the measurement $z_t$ for the current time step k is not yet available:

\begin{equation}

  p(x_t|z_{t-1}) = \int p(x_t| x_{t-1}) p(x_{t_1}|z_{t-1}) dx_{t1}

\end{equation}

\subsection{Fast SLAM}

In FastSLAM, alike in EKF SLAM, poses are assumed to behave according to a probabilist law name motion model with an
underlying density. FastSLAM decomposes the SLAM problem into one robot localization problem, and a collection of $K$
landmark estimation problems.

\subsection{FastSLAM 1.0 x FastSLAM 2.0}

Both algorith use the a low dimensional Extended Kalman Filter to estimate the posterior over the map features.\\
Basically, the only modification in the version 2 is that the proposal distribution should not only rely on the precious
estimate of the pose $s_{t-1}$ but also on the actual measurements $z_t$.\ref{Calonder2006}

\subsection{Grid-based FastSLAM}
Grid-based FastSLAM can model the environment using grid maps with out predefining any landmark position. Therefore, we
can now solve the SLAM problem in an arbitray environment.

\subsection{GraphSLAM}

A Graph-based SLAM approach construct a simplified estimation problem by abstracting the raw sensor measurements. These
raw measurements are replaced by the edges in the graph which can be seen as `virtual measurements'.\\
We can apply several algorithm to reduce the dimensionality of the optimization problem.\cite{Thrun2006a}

\begin{figure}[thpb]
      \centering
      \includegraphics[width=\linewidth]{graph.png}
      \caption{Illustration of the acquisition of the information matrix in GraphSLAM. The left diagram shows the
      dependence graph \cite{Thrun2006a}}
      \label{fig:graph}
\end{figure}

\subsection{RTAB-MAP}

Real-Time Appearance-Based Mapping is a RGB-D, Stereo and Lidar Graph-Based SLAM approach based on an incremental appearance-based loop closure detector. The loop closure detector uses a bag-of-words approach to determinate how likely a new image comes from a previous location or a new location. When a loop closure hypothesis is accepted, a new constraint is added to the map’s graph, then a graph optimizer minimizes the errors in the map. A memory management approach is used to limit the number of locations used for loop closure detection and graph optimization, so that real-time constraints on large-scale environnements are always respected. RTAB-Map can be used alone with a handheld Kinect, a stereo camera or a 3D lidar for 6DoF mapping, or on a robot equipped with a laser rangefinder for 3DoF mapping.

\section{Scene robot configuration}
\subsection{Robot Configuration}


The robot model was based in the robot from the previous project. The RGB camera was replaced with a RGB-D camera. The
LaserScan could be removed since we can create a scan topic using the depth point cloud from the RGB-D camera. See
Fig\ref{fig:rgbd}

\begin{figure}[thpb]
      \centering
      \includegraphics[width=\linewidth]{model.png}
      \caption{Robot with RGB-D Camera}
      \label{fig:rgbd}
\end{figure}

\subsection{Scene Configuration}

A new scene was builted using the Gazebo. We placed diferents objects in the scene to facilitate the loops closures. See
Fig\ref{fig:room}.

\begin{figure}[thpb]
      \centering
      \includegraphics[width=\linewidth]{newscene.png}
      \caption{Room world}
      \label{fig:room}
\end{figure}

\section{Results}

After navigating around the worlds, the 3D and 2D occupacy grid map was generated for each environment. 

\begin{figure}[thpb]
      \centering
      \includegraphics[width=\linewidth]{3dkitchen.png}
      \caption{3D Map from the Kitchen}
\end{figure}

\begin{figure}[thpb]
      \centering
      \includegraphics[width=\linewidth]{mapkitchen.png}
      \caption{Mapping from the Kitchen}
\end{figure}

\begin{figure}[thpb]
      \centering
      \includegraphics[width=\linewidth]{maproom.png}

      \caption{Mapping from the Room}
\end{figure}

\section{Discussion}

The mapping for the provided map as quite fast because there's many features in the map. In the first try in the created
map the robot didn't find many features and the mapping was not working. We add more object through the scene, after
that the mapping was quite fast.
In the created scene, there's a Armbot, which it's keep swinging, the SLAM was not able to map this part.
\section{Future work}

    In the future, deploy the model in a real world and perfome similar task performed here.
    Try different algorithms to the SLAM problem. Like Hector SLAM, ORB-SLAM, something similar to this article\cite{VehicularTechnologySociety2017}.


\nocite{*}
\bibliography{bib}
\bibliographystyle{ieeetr}

\end{document}

